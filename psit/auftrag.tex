\documentclass{article}
\usepackage[utf8]{inputenc}
\usepackage[numbers]{natbib}
\usepackage{amsmath}
\usepackage{url}

\author{Joris Damian Morger, IT12T}

\title{Zweikörpersysteme - Auftrag PSIT HS 2013}

\begin{document}
	\bibliographystyle{plain}
	\bibliography{auftrag}
	\maketitle


	\ldots and so it begins

	Wir betrachten im folgenden zwei isolierte Körper Erde und Mond mit den Massen $m$ (Mond) und $M$ (Erde), wobei der Mond in ellipsenförmig um die Erde rotiert.

	Per Definition:
	
	$m$ = $7.35 \times 10^{24}$ kg \cite{wiki1}

	$M$ = $5.79 \times 10^{24}$ kg \cite{wiki1}

	$d$ (Abstand) = $384'400$ km \cite{wiki2}


	% \includegraphics*[100,100][300,300]{situationsskizze}



\end{document}

