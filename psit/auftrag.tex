\documentclass{article}
\usepackage[utf8]{inputenc}
\usepackage[numbers]{natbib}
\usepackage{amsmath}
\usepackage{url}

\author{Joris Damian Morger, Nicolas Gagliani, IT12T}

\title{Zweikörpersysteme - Auftrag PSIT HS 2013}

\begin{document}
	\bibliographystyle{ieeetr}
	\bibliography{auftrag}
	\maketitle

	\section*{Intro}

	Wir betrachten im folgenden zwei isolierte Körper Erde und Mond mit den Massen $m_{Mond}$ und $M_{Erde}$, wobei der Mond in ellipsenförmig um die Erde rotiert.
	\par
	\bigskip
	Per Definition:
	$$m_{\text{M}} = 7.35 \times 10^{22} \text{ kg} \cite{wiki1}$$

	$$m_{\text{E}} = 5.79 \times 10^{24} \text{ kg} \cite{wiki1}$$

	$$r_{\text{ME}} (Abstand) = 3,844 \cdot 10^8 \text{ m} \cite{wiki2}$$

	$$\gamma = 6.6742e^{-22} \frac{m^{3} }{kg \cdot s^{2} }$$


	\section*{Teil A}
	\subsection{Gravitationskraft}
	Erde und Mond ziehen sich an. Die Kraft zwischen den beiden Massen lässt sich wie folgt berechnen:

	Wir wissen:

	$$T = 27,3217d \approx 2360594s$$
	$$\omega = \frac{2\pi}{T}$$
	$$F_{Z} = m\omega^2r$$

	Daher:
	$$\omega_\text{M} = \frac{2\pi}{T} = \frac{2\pi}{27,3217d} = \frac{2\pi}{2360594s} \approx 2.66\cdot10^{-6} s^{-1}$$
	$$F_\text{Z} = m_\text{M} \cdot \omega_\text{M}^2 \cdot r_\text{ME} \approx 2 \cdot 10^{20} N$$

	$$F_{G,M} = \gamma \cdot \frac{{m_\text{E} \cdot m_{\text{M}}}}{{{|{\vec{r_{ME}}}}|^2}}} \cdot \vec{n} \approx -2 \cdot 10^{20} N$$

	\subsection{Kreisbewegung}
	Für die Kreisbewegung gilt:
	$$\vec{r}(t) =
		\[ \left( \begin{array}{cc}
			r \cdot cos(\omega t)\\
			r \cdot sin(\omega t)\\
		\end{array} \right)\]
	$$

	$$\vec{v}(t) = \dot{\vec{r}}(t)
		\[ \left( \begin{array}{cc}
			-r \omega \cdot -sin(\omega t)\\
			r \omega \cdot cos(\omega t)\\
		\end{array} \right)\]
	$$

	$$\vec{a}(t) = \dot{\vec{v}}(t)
		\[ \left( \begin{array}{cc}
			-r \omega^2 \cdot cos(\omega t)\\
			-r \omega^2 \cdot sin(\omega t)\\
		\end{array} \right)\]
	$$

	% \includegraphics*[100,100][300,300]{situationsskizze}



\end{document}

