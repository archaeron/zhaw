\documentclass{article}
\usepackage[utf8]{inputenc}
\usepackage[numbers]{natbib}
\usepackage{amsmath}
\usepackage{url}

\author{Joris Damian Morger, IT12T}

\title{Zweikörpersysteme - Auftrag PSIT HS 2013}

\begin{document}
	\bibliographystyle{plain}
	\bibliography{auftrag}
	\maketitle
	

	\section*{Intro}	

	Wir betrachten im folgenden zwei isolierte Körper Erde und Mond mit den Massen $m_{Mond}$ und $M_{Erde}$, wobei der Mond in ellipsenförmig um die Erde rotiert.
	\par
	\bigskip
	Per Definition:
	$m_{Mond} = 7.35 \times 10^{24} kg$ \cite{wiki1}

	$m_{Erde} = 5.79 \times 10^{24} kg$ \cite{wiki1}

	$d_{MondErde} (Abstand) = 384 400 km$ \cite{wiki2}

	$g = 6,6742e^{-22} \frac{m^{3} }{kg*s^{2} }$ 
	\section*{Teil A}

	Erde und Mond ziehen sich an. Die Kraft zwischen den beiden Massen lässt sich wie folgt berechnen:
	\par
	\bigskip
	${F_{G,M}} = g \cdot \frac{{{m_Erde} \cdot {m_Mond}}}{{{d_{MondErde}}}^2}} = todo$


	\par
	\bigskip

	% \includegraphics*[100,100][300,300]{situationsskizze}



\end{document}

